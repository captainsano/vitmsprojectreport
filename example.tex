%
% VIT Final Year Project Report.
%
% Author: Santhos Baala RS, VIT University (09MSE038)
%
\documentclass{vitmsprojectreport}

\begin{document}

\pagenumbering{roman} % Set page numbering to roman style

%----------------------------%
%         Title Page         %
%----------------------------%
\maketitlepage%
{Using~\LaTeX~for~High~Quality~\\Project~Report~Generation}%
{Santhos~Baala~RS}%
{09MSE038}%   Prepend reg no with \\, in case your name is long
{Dr.~Krishna~Chandramouli}%
{Associate~Professor}%

%----------------------------%
%      Declaration Page      %
%----------------------------%
\makedeclarationpage%

%----------------------------%
%       Bonafide Page        %
%----------------------------%
\makebonafidepage%

%----------------------------%
%    Acknowledgements Page   %
%----------------------------%
\makeackpage%
{Dr.~Sankaran}%
{Prof.~Jayaram Reddy A}%
{Dr.~Krishna Chandramouli}%

%----------------------------%
%      Exec. Summary Page    %
%----------------------------%
\makeexecsummarypage{
This document describes how to use the vitmsprojectreport class with \LaTeX\ to produce high quality typeset project report that is suitable for submission to the School of Information Technology and Engineering (SITE). The class can further be extended to various courses and department by modifying the title page and department information.
}

\tableofcontents  % Table Of Contents
\clearpage
\listoffigures    % List of Figures
\clearpage
\listoftables     % List of Tables
\clearpage

% Reset page counter for contents and 
% change style to arabic
\setcounter{page}{1}
\pagenumbering{arabic}

%======== Contents Start Here ==========%

%----- INTRODUCTION -------%
\chapter{Introduction}

\section{About the Template}

With a basic understanding of the \LaTeX\ language and say 10 to 20 commands, an author can produce beautiful typeset project report quickly, with minimal effort. The purpose of this document is to serve as a user guide for the project report template and document its special behaviours. Examples have also been provided for common tasks such as insertion of images, table, equation, etc., that can be copied as is by the users. It is assumed that the user has a basic knowledge on working of the \LaTeX\ system. Those lacking are strongly urged to lookup some excellent literature through~\cite{kottwitz2011latex}. 

Sufficient examples have been given in the document and as users request solutions for specific problems that they encounter, more will be added. However, an understanding of the \LaTeX\ system will help the user in unexplainable ways\footnote{The \LaTeX\ system is a vast ocean. That said, you can mostly get away with copy-paste skills. You have to observe the source code of the examples and learn}. The main advantage of using the custom class is that, the user need not worry about the final layout. The class may be changed during the development of the project report, the user needs to just place the latest version of the class file, without touching the content\footnote{It is always recommended to get the latest class file from the repository and compile before project report submission.}. \LaTeX, along with the custom class file gives an incredible expressive power its users so that they can focus on the semantics of the document.

\section{Generic \LaTeX\ Commands}

\LaTeX\ contains many commands and with it comes numerous built-in packages that add many features and provide options to customise the output. However, the user is advised to stick to the basic commands, illustrated in this document so that unexpected or incorrect output can be avoided. If, however a specific feature is requested it shall be considered and incorporated into the template. The class file for project report has been extended from the standard \emph{report} class, supplied by \LaTeX. Therefore, whatever applied for report also applies for the custom class file.

\section{Generating the Final Document}

The template development is an ongoing process. It is mature at this point from the point of view of semantics. Layout or default contents (e.g bonafide, acknowledgement, etc.,) may change at a later point in time. Therefore, before generating the final document, please make sure to check out a frozen version of the class file from the repository before compiling the class file.

\section{Asking Questions}

You can post your questions in the forum or if you encounter any problem related to unexpected/incorrect output due to the template send an e-mail to the contributors at the github page.

\section{Contributing}

The project is hosted on GitHub~\cite{github2014} and uses the GIT repository management system. Contributors can fork the project and send pull requests to the repository for the changes to be merged. The patch will be evaluated and if found to be good, merged into the master branch of the repository. If however, a separate template is required for other departments and courses, the project can be forked and maintained separately. Issues, pertaining to the template, such as rendering faults, can be posted in the issues section of the repo. The contributors shall also put up a list of items that need improvement or new features in progress that people can contribute to.

%----- Auto-Generated Pages -------%
\chapter{Auto-Generated Pages}

The parts of the document, unique to the project report are generated by a predefined template, using simple commands. The order in which the commands are issued determine the corresponding order of these individual pages. The arguments that need to be passed to these commands are explained in the following sections. The sequence of commands to generate such pages have already been placed in the example document and it is recommended not to touch those, expect when inserting your project specific information.

\section{The Title Page}

The user can generate the title page using the \verb+\maketitlepage+ command. The detailed syntax is given in \figurename~\ref{fig:maketitlepage_syntax}.

\begin{figure}[htb]
\centering
\hrulefill
\begin{verbatim}
\maketitlepage
{Project Title}
{Name}
{RegNo}
{Guide Name}
{Guide Title} 
\end{verbatim}
\hrulefill
\caption{The syntax of \textbackslash maketitlepage command}
\label{fig:maketitlepage_syntax}
\end{figure}

Note that you should always set the title page to be the first page of the document, so issue this command right after the document begins. The template also collects information like your name and regno when you declare the title page so that it can be auto-inserted into other pages like declaration and bonafide. Long names will automatically push the register number to the next line. However, to be safe, prepend your RegNo with \textbackslash\textbackslash, for e.g \verb+{\\09MSE038}+, only if your name is long.

\section{Declartion Page}

TODO

\section{Bonafide Page}

TODO

\section{Acknowledgement Page}

TODO

\section{Executive Summary Page}

TODO

%====== Auto-Insert Chapter for References =======
\insertreferences{Example}
%=================================================

\end{document}